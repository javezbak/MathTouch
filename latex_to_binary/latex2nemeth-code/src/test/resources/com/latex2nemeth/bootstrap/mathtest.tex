\documentclass[twoside,a4paper,leqno,11pt]{book}
\usepackage[greek]{babel}
\usepackage[utf8x]{inputenc}

\usepackage{srcltx}

\usepackage{latexsym} 

\usepackage{amsmath}

\usepackage{amssymb} 



%%%%%%%%%%%%%%%%%%%%%%%%%%%%%%%%%%%%%%%%%%%%%%%%%%%%%%%%%%


%%%%%%%%%%%% New theorems %%%%%%%%%%%%%%%%%%%%%%%%
\newtheorem{theorem}{Θεώρημα}[section]
\newtheorem{lemma}[theorem]{Λήμμα}
\newtheorem{proposition}[theorem]{Πρόταση}
\newtheorem{application}[theorem]{Εφαρμογή}
\newtheorem{corollary}[theorem]{Πόρισμα}
\newtheorem{definition}[theorem]{Ορισμός}
\newtheorem{exercise}[theorem]{Άσκηση}
\newtheorem{example}[theorem]{Παράδειγμα}
\newtheorem{examples}[theorem]{Παραδείγματα}
\newtheorem{problem}[theorem]{Πρόβλημα}
\newtheorem{remark}[theorem]{Παρατήρηση}
\newtheorem{remarks}[theorem]{Παρατηρήσεις}
%%%%%%%%%%%%%%%%%%%%%%%%%%%%%%%%%%%%%%%%%%%%%%%%%%%


%%%%%%%%%%%%%%%%%%%%% Document starts %%%%%%%%%%%%
\begin{document}
%%%%%%%%%%%%%%%%%%%%%%%%%%%%%%%%%%%%%%%%%%%%%%%%%%


\textbf{Απειροστικός Λογισμός ΙΙ}
\textbf{Πρόχειρες Σημειώσεις}
\textbf{Τμήμα Μαθηματικών}
\textbf{Πανεπιστήμιο Αθηνών}
\textbf{2010--11}

%%%%%%%%%%%%%%%%%%%%%%%%%%%%%%%%%%%%%%%%%%%%%%%%%%%%%%%%%%%%%%%%%%%%%%%%%%%%%%%%%%
\chapter{Υπακολουθίες και βασικές ακολουθίες}
%%%%%%%%%%%%%%%%%%%%%%%%%%%%%%%%%%%%%%%%%%%%%%%%%%%%%%%%%%%%%%%%%%%%%%%%%%%%%%%%%%


\section{Υπακολουθίες}

\begin{definition} \upshape Έστω $(a_n)$ μια ακολουθία πραγματικών αριθμών.
Η ακολουθία $(b_n)$ λέγεται \textit{υπακολουθία} της $(a_n)$ αν υπάρχει
γνησίως αύξουσα ακολουθία φυσικών αριθμών $k_1 < k_2< \cdots < k_n <
k_{n+1}<\cdots $ ώστε
$$b_n = a_{k_n}\;\hbox{ για κάθε }\;n \in {\mathbb N}.\leqno (1.1.1)$$
Με άλλα λόγια, οι όροι της $(b_n)$ είναι οι $a_{k_1}, a_{k_2},
\ldots, a_{k_n}, \ldots $, όπου $k_1 < k_2< \cdots < k_n <
k_{n+1}<\cdots$. Γενικά, μια ακολουθία έχει πολλές (συνήθως άπειρες
το πλήθος) διαφορετικές υπακολουθίες.
\end{definition}


%%%%%%%%%%%%%%%%%%%%%%%%%%%%%%%%%%%%%%%%%%%%%%%%%%%%%%%%%%%%%%%%%%%%%%%%%%%%%
\chapter{Σειρές πραγματικών αριθμών}
%%%%%%%%%%%%%%%%%%%%%%%%%%%%%%%%%%%%%%%%%%%%%%%%%%%%%%%%%%%%%%%%%%%%%%%%%%%%%%%


\section{Σύγκλιση σειράς}

\begin{definition} \upshape Έστω $(a_k)$ μια ακολουθία πραγματικών
αριθμών. Θεωρούμε την ακολουθία $$s_n=a_1+\cdots +a_n.\leqno
(2.1.1)$$ Δηλαδή,
$$s_1=a_1,\  s_2=a_1+a_2,\  s_3=a_1+a_2+a_3,\  \ldots \leqno (2.1.2)$$
Το σύμβολο $\sum_{k=1}^{\infty }a_k$ είναι η \textit{σειρά} με
$k$-οστό όρο τον $a_k$. Το άθροισμα $s_n=\sum_{k=1}^na_k$
είναι το \textit{$n$-οστό μερικό άθροισμα} της σειράς
$\sum_{k=1}^{\infty }a_k$ και η $(s_n)$ είναι η {\it
ακολουθία των μερικών αθροισμάτων} της σειράς $ \sum_{k =
1}^{\infty }a_k$.

Αν η $(s_n)$ συγκλίνει σε κάποιον πραγματικό αριθμό $s$, τότε
γράφουμε
$$s = a_1 + a_2 + \cdots + a_n + \cdots\  \hbox{ή}\  s=\sum_{k=1}^{\infty }a_k\leqno (2.1.3)$$
και λέμε ότι η σειρά \textit{συγκλίνει} (στο $s$), το δε όριο
$s=\lim_{n\to\infty }s_n$ είναι το \textit{άθροισμα} της σειράς.
\end{definition}


%%%%%%%%%%%%%%%%%%%%%%%%%%%%%%%%%%%%%%%%%%%%%%%%%%%%%%%%%%%%%%%%%%%%%%%%%%%%%%%%%%%%%%%
\chapter{Ολοκλήρωμα \textlatin{Riemann}}
%%%%%%%%%%%%%%%%%%%%%%%%%%%%%%%%%%%%%%%%%%%%%%%%%%%%%%%%%%%%%%%%%%%%%%%%%%%%%%%%%%%%%%%%

\section{Ο ορισμός του \textlatin{Darboux}}

Σε αυτήν την παράγραφο δίνουμε τον ορισμό του ολοκληρώματος
\textlatin{Riemann} για \textbf{φραγμένες} συναρτήσεις που ορίζονται σε ένα
κλειστό διάστημα. Για μια φραγμένη συνάρτηση $f:[a,b]\to {\mathbb
R}$ με μη αρνητικές τιμές, θα θέλαμε το ολοκλήρωμα να δίνει το
εμβαδόν του χωρίου που περικλείεται ανάμεσα στο γράφημα της
συνάρτησης, τον οριζόντιο άξονα $y=0$ και τις κατακόρυφες ευθείες
$x=a$ και $x=b$.

\begin{definition} \upshape (α) Έστω $[a,b]$ ένα κλειστό διάστημα.
\textbf{Διαμέριση} του $[a,b]$ θα λέμε κάθε πεπερασμένο υποσύνολο
$$P
=\{ x_0,x_1,\ldots ,x_n\}\leqno (4.1.1)$$ του $[a,b]$ με $x_0=a$
και $x_n=b$. Θα υποθέτουμε πάντα ότι τα $x_k\in P $ είναι
διατεταγμένα ως εξής:
$$a=x_0<x_1<\cdots <x_k<x_{k+1}<\cdots <x_n=b.\leqno (4.1.2)$$
Θα γράφουμε
$$P =\{ a=x_0<x_1<\cdots <x_n=b\}\leqno (4.1.3)$$ για να τονίσουμε αυτήν
ακριβώς τη διάταξη. Παρατηρήστε ότι από τον ορισμό, κάθε διαμέριση
$ P $ του $[a,b]$ περιέχει τουλάχιστον δύο σημεία: το $a$ και το
$b$ (τα άκρα του $[a,b]$).



 (β) Κάθε διαμέριση $ P =\{ a=x_0<x_1<\cdots <x_n=b\}$
χωρίζει το $[a,b]$ σε $n$ υποδιαστήματα $[x_k,x_{k+1}]$,
$k=0,1,\ldots ,n-1$. Ονομάζουμε \textbf{πλάτος} της διαμέρισης $ P $
το μεγαλύτερο από τα μήκη αυτών των υποδιαστημάτων. Δηλαδή, το
πλάτος της διαμέρισης ισούται με
$$\| P\|:=\max\{ x_1-x_0,x_2-x_1,\ldots ,x_n-x_{n-1}\}.\leqno (4.1.4)$$
Παρατηρήστε ότι δεν απαιτούμε να ισαπέχουν τα $x_k$ (τα $n$
υποδιαστήματα δεν έχουν απαραίτητα το ίδιο μήκος).



 (γ) Η διαμέριση $ P_1$ λέγεται \textbf{εκλέπτυνση} της $ P
$ αν $ P \subseteq P_1$, δηλαδή αν η $P_1$ προκύπτει από την $ P $
με την προσθήκη κάποιων (πεπερασμένων το πλήθος) σημείων. Σε αυτήν
την περίπτωση λέμε επίσης ότι η $ P_1$ είναι \textit{λεπτότερη} από
την $ P $.



 (δ) Έστω $ P_1, P_2$ δύο διαμερίσεις του $[a,b]$. Η 
\textbf{κοινή εκλέπτυνση} των $ P_1, P_2$ είναι η διαμέριση $ P = P_1\cup
P_2$. Εύκολα βλέπουμε ότι η $ P $ είναι διαμέριση του $[a,b]$ και
ότι αν $ P^{\prime }$ είναι μια διαμέριση λεπτότερη τόσο από την $
P_1$ όσο και από την $ P_2$ τότε $ P^{\prime }\supseteq P $
(δηλαδή, η $ P = P_1\cup P_2$ είναι η μικρότερη δυνατή διαμέριση
του $[a,b]$ που εκλεπτύνει ταυτόχρονα την $ P_1$ και την $ P_2$).
\end{definition}


\section{Ιδιότητες του ολοκληρώματος \textlatin{Riemann}}

Σε αυτή την παράγραφο αποδεικνύουμε αυστηρά μερικές από τις πιο
βασικές ιδιότητες του ολοκληρώματος \textlatin{Riemann}. Οι αποδείξεις
των υπολοίπων είναι μια καλή άσκηση που θα σας βοηθήσει να
εξοικειωθείτε με τις διαμερίσεις, τα άνω και κάτω αθροίσματα κλπ.

\begin{theorem}
Αν $f(x)=c$ για κάθε $x\in [a,b]$, τότε
$$\int_a^bf(x)dx =c(b-a).\leqno (4.4.1)$$
\end{theorem}



%%%%%%%%%%%%%%%%%%%%%%%%%%%%%%%%%%%%%%%%%%%%%%%%%%%%%%%%%%%%%%%%%%%%%%%%%%%%%%%%%%%%%%%%%%%
\chapter{Τεχνικές ολοκλήρωσης}
%%%%%%%%%%%%%%%%%%%%%%%%%%%%%%%%%%%%%%%%%%%%%%%%%%%%%%%%%%%%%%%%%%%%%%%%%%%%%%%%%%%%%%%%%%%

Σε αυτό το Κεφάλαιο περιγράφουμε, χωρίς ιδιαίτερη αυστηρότητα, τις
βασικές μεθόδους υπολογισμού ολοκληρωμάτων. Δίνεται μια συνάρτηση
$f$ και θέλουμε να βρούμε μια αντιπαράγωγο της $f$, δηλαδή μια
συνάρτηση $F$ με την ιδιότητα $F^{\prime }=f$. Τότε,
$$\int f(x)dx =F(x)+c.$$

\section{Ολοκλήρωση με αντικατάσταση}

\subsection{Πίνακας στοιχειωδών ολοκληρωμάτων}

Κάθε τύπος παραγώγισης $F^{\prime }(x)=f(x)$ μας δίνει έναν τύπο
ολοκλήρωσης: η $F$ είναι αντιπαράγωγος της $f$. Μπορούμε έτσι να
δημιουργήσουμε έναν πίνακα βασικών ολοκληρωμάτων, αντιστρέφοντας
τους τύπους παραγώγισης των πιο βασικών συναρτήσεων:
\begin{eqnarray*}
\int x^adx =\frac{x^{a+1}}{a+1},\qquad a\neq -1, &&
\int\frac{1}{x}\,dx = \ln |x| +c\\
\int e^xdx = e^x+c, &&
\int\sin x\,dx = -\cos x+c\\
\int\cos x\,dx = \sin x+c, &&
\int\frac{1}{\cos^2x}\,dx = \tan x+c\\
\int\frac{1}{\sin^2x}\,dx = -\cot x+c , &&
\int\frac{1}{\sqrt{1-x^2}}\,dx = \arcsin x+c\\
\int\frac{1}{1+x^2}\,dx =\arctan x+c. &&
\end{eqnarray*}


\section{Ολοκλήρωση ρητών συναρτήσεων}

Σε αυτή την παράγραφο περιγράφουμε μια μέθοδο με την οποία μπορεί
κανείς να υπολογίσει το αόριστο ολοκλήρωμα οποιασδήποτε ρητής
συνάρτησης
$$f(x)=\frac{p(x)}{q(x)}=\frac{a_nx^n+a_{n-1}x^{n-1}+\cdots
+a_1x+a_0}{b_mx^m+b_{m-1}x^{m-1}+\cdots +b_1x+b_0}.\leqno (6.3.1)$$
Η πρώτη παρατήρηση είναι ότι μπορούμε πάντα να υποθέτουμε ότι $n<m$.
Αν ο βαθμός $n$ του αριθμητή $p(x)$ είναι μεγαλύτερος ή ίσος από τον
βαθμό $m$ του παρονομαστή $q(x)$, τότε διαιρούμε το $p(x)$ με το
$q(x)$: υπάρχουν πολυώνυμα $\pi (x)$ και $\upsilon (x)$ ώστε ο
βαθμός του $\upsilon (x)$ να είναι μικρότερος από $m$ και $$p(x)=\pi
(x)q(x)+\upsilon (x).\leqno (6.3.2)$$ Τότε,
$$f(x)=\frac{\pi (x)q(x)+\upsilon (x)}{q(x)}=\pi (x)+\frac{\upsilon
(x)}{q(x)}.\leqno (6.3.3)$$ Συνεπώς, για τον υπολογισμό του $\int
f(x)\,dx$ μπορούμε τώρα να υπολογίσουμε χωριστά το $\int \pi
(x)\,dx$ (απλό ολοκλήρωμα πολυωνυμικής συνάρτησης) και το
$\int\frac{\upsilon (x)}{q(x)}\,dx$ (ρητή συνάρτηση με την πρόσθετη
ιδιότητα ότι $\mathrm{deg}(\upsilon )<\mathrm{deg}(q)$).

Υποθέτουμε λοιπόν στη συνέχεια ότι $f=p/q$ και $\mathrm{deg}(p)<
\mathrm{deg}(q)$. Μπορούμε επίσης να υποθέσουμε ότι $a_n=b_m=1$.
Χρησιμοποιούμε τώρα το γεγονός ότι κάθε πολυώνυμο αναλύεται σε
γινόμενο πρωτοβάθμιων και δευτεροβάθμιων όρων. Το $q(x)=x^m+\cdots
+b_1x+b_0$ γράφεται στη μορφή
$$q(x)=(x-\alpha_1)^{r_1}\cdots
(x-\alpha_k)^{r_k}(x^2+\beta_1x+\gamma_1)^{s_1}\cdots
(x^2+\beta_lx+\gamma_l)^{s_l}.\leqno (6.3.4)$$ Οι $\alpha_1,\ldots
,\alpha_k$ είναι οι πραγματικές ρίζες του $q(x)$ (και $r_j$ είναι η
πολλαπλότητα της ρίζας $\alpha_j$) ενώ οι όροι
$x^2+\beta_ix+\gamma_i$ είναι τα γινόμενα
$(x-z_i)(x-\overline{z_i})$ όπου $z_i$ οι μιγαδικές ρίζες του $q(x)$
(και $s_i$ είναι η πολλαπλότητα της ρίζας $z_i$). Παρατηρήστε ότι
κάθε όρος της μορφής $x^2+\beta_ix+\gamma_i$ έχει αρνητική
διακρίνουσα. Επίσης, οι $k,s\geq 0$ και $r_1+\cdots +r_k+2s_1+\cdots
+2s_l=m$ (ο βαθμός του $q(x)$).

Γράφουμε την $f(x)$ στη μορφή
$$f(x)=\frac{x^n+a_{n-1}x^{n-1}+\cdots +a_1x+a_0}{(x-\alpha_1)^{r_1}\cdots
(x-\alpha_k)^{r_k}(x^2+\beta_1x+\gamma_1)^{s_1}\cdots
(x^2+\beta_lx+\gamma_l)^{s_l}},\leqno (6.3.5)$$ και την ((αναλύουμε
σε απλά κλάσματα)): υπάρχουν συντελεστές $A_{jt}$,
$B_{it},\Gamma_{it}$ ώστε
\begin{eqnarray*}
f(x) &=&
\frac{A_{11}}{x-\alpha_1}+\frac{A_{12}}{(x-\alpha_1)^2}+\cdots
+\frac{A_{1r_1}}{(x-\alpha_1)^{r_1}}\\
&& +\cdots \\
&& +\frac{A_{k1}}{x-\alpha_k}+\frac{A_{k2}}{(x-\alpha_k)^2}+\cdots
+\frac{A_{kr_1}}{(x-\alpha_k)^{r_k}}\\
&&
+\frac{B_{11}x+\Gamma_{11}}{x^2+\beta_1x+\gamma_1}+\frac{B_{12}x+\Gamma_{12}}{(x^2+\beta_1x+\gamma_1)^2}+\cdots
+\frac{B_{1s_1}x+\Gamma_{1s_1}}{(x^2+\beta_1x+\gamma_1)^{s_1}}\\
&& +\cdots \\
&&
+\frac{B_{l1}x+\Gamma_{l1}}{x^2+\beta_lx+\gamma_l}+\frac{B_{l2}x+\Gamma_{l2}}{(x^2+\beta_lx+\gamma_l)^2}+\cdots
+\frac{B_{ls_1}x+\Gamma_{ls_l}}{(x^2+\beta_lx+\gamma_l)^{s_l}}.
\end{eqnarray*}


\end{document}
